\chapter{Introduction}\label{chap:intro}

% \vspace{-5pt}

The Standard Model (SM)~\cite{pdg} has satisfied experimental results in particle physics for 100 years. Many searches in particle physics seek to better understand this standard model, since it does not include gravity or Dark Matter. This dissertation focuses on searches for theoretical particles decaying to SM $t\bar{t}$ pairs.


The energy frontier of particle physics occurs at the Large Hadron Collider (LHC) in Geneva, Switzerland, the highest energy collider in history.  In this analysis, we search for a heavy  $t\bar{t}$ resonances decaying from particles predicted by BSM theories. Some of these BSM models predict extra dimensions, resulting in Randall-Sundrum Kaluza-Klein gluons~\cite{rs1, ExtraDim}. We also look for $Z'$ resonances in leptophobic topcolor models \cite{leptophobicZprime, Zprimettxs, ZprimeCoupledtoGen3, WarpedGaugeBosons}, as well as dark matter (DM) $Z'$ particles \cite{wasmer_dark_matter}.

Chapter 2 discusses the established particle physics theories. Chapter 3 descrinesn the CMS detector whose data the analysis was performed with. Chapter 4 describes how the data from the detector is used to identify particles. Chapter 5 describes the analysis.


In the past, searches at CDF and D0 at the Tevatron, and CMS and ATLAS at the Large Hadron Collider (LHC) have set limits for heavy $t\bar{t}$ resonances. Experiments conducted at the Tevatron searched for masses up to 900 GeV \cite{cdftt3, d0tt} and experiments at the LHC have searched for masses up to 4 TeV, most recently using Run 2 data at $\sqrt{s} = 13$ TeV \cite{7tevZprime_CMSAllHad, 7tevZprime_CMSSemilept, 7tevZprime_ATLASAllHad, 7tevZprime_ATLASSemilept, 8tevZprime_CMSAllHad, 8tevZprime_CMSAllHadSemilept, 8tevZprime_CMSAllHadSemileptLept, 8tevZprime_ATLASSemilept, AN-2016-459}. The CMS and ATLAS searches are categorized by the decay of the W in the $t \rightarrow W b$ decay into all-hadronic, semi-leptonic, and fully-leptonic final states. For this analysis, only the all-hadronic channel is considered. The other two channels of a $t\bar{t}$ decay - the semileptonic and dileptonic channels - will be part of a fully combined analysis in the future. An example of the decay of one of the theoretical particles is shown in the Feynman diagram in Figure~\ref{fig:feynman_ttbar}.


\begin{figure}[h]
	\centering
	\begin{tikzpicture}
		\begin{feynman}
			\vertex (a);
			\vertex [above right=of a] (t1);
			\vertex [below right=of a] (t2);
			\vertex [left=3cm of a] (g);
			
			\vertex [above left=of g] (e) {\(q\)};
			\vertex [below left=of g] (f) {\(\overline q\)};
			
			\vertex [below right=0.25cm and 1cm of t1] (f1) {\(b\)};
			\vertex [above right=0.5cm and 2cm of t1] (c);
			\vertex [above right=0.5cm and 1cm of c] (f2) {\(\overline q\)};
			\vertex [below right=0.5cm and 1cm of c] (f3) {\(q\)};
			
			\vertex [above right=0.25cm and 1cm of t2] (f4) {\(\overline b\)};
			\vertex [below right=0.5cm and 2cm of t2] (cc);
			\vertex [above right=0.5cm and 1cm of cc] (f5) {\(q\)};
			\vertex [below right=0.5cm and 1cm of cc] (f6) {\(\overline q\)};
			
			\diagram* {
				(e) -- [fermion] (g),
				(f) -- [anti fermion] (g),
				(g) -- [boson, edge label=\({g_{KK},Z^{\prime}}\)] (a),
				(a) -- [fermion, edge label=\(t\)] (t1),
				(a) -- [anti fermion, edge label'=\(\overline t\)] (t2),
				(t1) -- [fermion] (f1),
				(t1) -- [boson, edge label=\(W^{+}\)] (c),
				(c) -- [anti fermion] (f2),
				(c) -- [fermion] (f3),
				(t2) -- [anti fermion] (f4),
				(t2) -- [boson, edge label'=\(W^{-}\)] (cc),
				(cc) -- [fermion] (f5),
				(cc) -- [anti fermion] (f6),
			};
		\end{feynman}
	\end{tikzpicture}
	\caption{Decay of an exotic particle to \(t\bar{t}\) pair which decays hadronically.}
	\label{fig:feynman_ttbar}
\end{figure}
