\chapter{Introduction}\label{chap:intro}

% \vspace{-5pt}


The Standard Model (SM)~\cite{pdg} is the current best description of fundamental physics, having satisfied experimental particle physics results for decades. However, it is not a complete description, lacking gravity, Dark Matter (DM), and other open physical questions. Searches in particle physics seek to understand the model better and perhaps discover new particles to explain these missing phenomena. The dissertation focuses on searches for theoretical particles decaying to SM $t\bar{t}$ pairs.


The current analysis is performed using data from the CMS detector of the Large Hadron Collider (LHC) in Geneva, Switzlerland, the highest energy collider in history. In this analysis, we search for heavy resonances from particles predicted by Beyond the Standard Model (BSM) theories decaying to $t\bar{t}$ pairs. Some of these BSM models predict extra dimensions, resulting in Randall-Sundrum Kaluza-Klein gluons~\cite{rs1, ExtraDim}. We also look for $Z'$ resonances in leptophobic topcolor models \cite{leptophobicZprime, Zprimettxs, ZprimeCoupledtoGen3, WarpedGaugeBosons}, as well as DM $Z'$ particles \cite{wasmer_dark_matter}.


In the past, searches at the Tevatron by CDF and D0 collaborations, and at the LHC by CMS and ATLAS collaborations have set limits for heavy $t\bar{t}$ resonances. Experiments conducted at the Tevatron searched for masses up to 900 GeV \cite{cdftt3, d0tt} and experiments at the LHC have searched for masses up to 4 TeV, most recently using Run 2 data at $\sqrt{s} = 13$ TeV \cite{7tevZprime_CMSAllHad, 7tevZprime_CMSSemilept, 7tevZprime_ATLASAllHad, 7tevZprime_ATLASSemilept, 8tevZprime_CMSAllHad, 8tevZprime_CMSAllHadSemilept, 8tevZprime_CMSAllHadSemileptLept, 8tevZprime_ATLASSemilept, AN-2016-459}. The CMS and ATLAS searches are categorized by the decay of the W in the $t \rightarrow W b$ decay into all-hadronic, semi-leptonic, and fully-leptonic final states. For the current analysis, only the all-hadronic channel is considered. The other two channels of a $t\bar{t}$ decay - the semileptonic and dileptonic channels - will be part of an upcoming combined analysis. An example of the decay of one of the theoretical particles is shown in the Feynman diagram in Figure~\ref{fig:feynman_ttbar}.

Chapter 2 discusses the established particle physics theories, and gives more detail on our signals of interest. Chapter 3 describes the CMS detector whose data the analysis was performed with. Chapter 4 describes how the data from the detector is used to reconstruct the physics objects that are analyzed. Chapter 5 describes the analysis strategy and results.



\begin{figure}[h]
	\centering
	\begin{tikzpicture}
		\begin{feynman}
			\vertex (a);
			\vertex [above right=of a] (t1);
			\vertex [below right=of a] (t2);
			\vertex [left=3cm of a] (g);
			
			\vertex [above left=of g] (e) {\(q\)};
			\vertex [below left=of g] (f) {\(\overline q\)};
			
			\vertex [below right=0.25cm and 1cm of t1] (f1) {\(b\)};
			\vertex [above right=0.5cm and 2cm of t1] (c);
			\vertex [above right=0.5cm and 1cm of c] (f2) {\(\overline q\)};
			\vertex [below right=0.5cm and 1cm of c] (f3) {\(q\)};
			
			\vertex [above right=0.25cm and 1cm of t2] (f4) {\(\overline b\)};
			\vertex [below right=0.5cm and 2cm of t2] (cc);
			\vertex [above right=0.5cm and 1cm of cc] (f5) {\(q\)};
			\vertex [below right=0.5cm and 1cm of cc] (f6) {\(\overline q\)};
			
			\diagram* {
				(e) -- [fermion] (g),
				(f) -- [anti fermion] (g),
				(g) -- [boson, edge label=\({g_{KK},Z^{\prime}}\)] (a),
				(a) -- [fermion, edge label=\(t\)] (t1),
				(a) -- [anti fermion, edge label'=\(\overline t\)] (t2),
				(t1) -- [fermion] (f1),
				(t1) -- [boson, edge label=\(W^{+}\)] (c),
				(c) -- [anti fermion] (f2),
				(c) -- [fermion] (f3),
				(t2) -- [anti fermion] (f4),
				(t2) -- [boson, edge label'=\(W^{-}\)] (cc),
				(cc) -- [fermion] (f5),
				(cc) -- [anti fermion] (f6),
			};
		\end{feynman}
	\end{tikzpicture}
	\caption{Production of exotic particles decaying to \(t\bar{t}\) pair which then decays hadronically.}
	\label{fig:feynman_ttbar}
\end{figure}
