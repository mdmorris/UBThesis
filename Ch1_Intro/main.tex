\chapter{Introduction}\label{chap:intro}

% \vspace{-5pt}
\section{Motivation}\label{sec:ch1:intro}

The standard model has satisfied experimental results in particle physics for 100 years. Many searches in particle physics seek to better understand this standard model, and to analyze theories that expand on it.

With the discovery of the electron in 1897, the field of elementary particle physics began to grow. The electron was the first of what we consider now to be fundamental particles - particles that have no known substructure or excited states. In the early 1900s, Rutherford, Marsden, and Geiger discovered that atoms have nuclei in the center of the atom. For this experiment, Geiger also developed the Geiger counter, as originally he had to count the alpha particles scattering onto the gold foil by measuring flashes of light by eye.

Radioactivity was discovered before the particles responsible for radiation were discovered. In 1896, Becquerel discovered radiation of “beta rays”, now known to be electrons. The gamma ray, or photon, was theorized by Planck to explain black body radiation, and alpha rays, used in the gold foil experiment to bombard nuclei, are bound states of two protons and two neutrons. 

Elementary particle physics also gained the name “high energy experimental” physics, as scattering experiments became more common in the 1950s. These scattering experiments led to the discovery of hundreds of particles with short life spans. These new particles lacked a theory to explain them, or to predict future similar particles. Murray Gell-Mann developed a “quark model” theory to explain the new particles as “hadrons”, or bound states of the at the time three known quarks - the up, down, and strange quarks. Quarks were so named from a line in a James Joyce novel that contained the phrase “three quarks”. Three additional quarks have since been discovered.

In particle physics, spin is measured in units of $\bar{h}$. Electrons are spin ½ particles, and belong to a group called “fermions”, all with spin ½ particles. There are three generations of fundamental particles, with the first generation consisting of the up quark, down quark, electron, and electron neutrino. The first particle of generation greater than the first was the muon. Initially, it was mistaken for the pion, as the pion has a mass of 135 MeV and the muon a mass of 105.7 MeV. Further experiments showed that the new, higher generation particle did not interact hadronically, and so it must have been the second generation of the electron.

The fundamental forces in the Standard Model are the electromagnetic, weak, and strong forces. Gravity, which is weaker than the weak force by several orders of magnitude, does not have a place in our current Standard Model. Many searches in the current field of particle physics seek to unify our understanding of gravity with the Standard Model. 

The electromagnetic and weak forces are united under the electroweak theory, which is a generalization of Quantum Electrodynamics (QED).

The force particles are guage bosons - spin 1 bosons that act as force carriers for the three fundamental forces. The massless photon carries the electromagnetic force. Gluons, of which there are 8 massless and neutral versions, carry the strong force. The weak force is carried by three particles - $W^+$, $W^-$, and $Z^0$. 

\newpage

\section{$t\bar{t}$ Resonance Searches}



The standard model (SM) of particle physics explains electroweak symmetry breaking, the mechanism by which the $Z$ and $W$ bosons become massive. The reason why this scale is much different from the Planck scale (the ``hierarchy problem'') is still unknown. The contributions to the Higgs boson mass that involve top quark loops would diverge if there is not another mechanism to counter them in other beyond-SM (BSM) scenarios. In this analysis, we search for a heavy  $t\bar{t}$ resonances decaying from particles predicted by BSM theories. Some of these BSM models predict extra dimensions, resulting in Randall-Sundrum Kaluza-Klein gluons \cite{ExtraDim}. We also look for $Z'$ resonances in leptophobic topcolor models \cite{leptophobicZprime, Zprimettxs, ZprimeCoupledtoGen3, WarpedGaugeBosons}, as well as dark matter (DM) $Z'$ particles \cite{wasmer_dark_matter}.

In the past, searches at CDF and D0 at the Tevatron, and CMS and ATLAS at the Large Hadron Collider (LHC) have set limits for heavy $t\bar{t}$ resonances. Experiments conducted at the Tevatron searched for masses up to 900 GeV \cite{cdftt3, d0tt} and experiments at the LHC have searched for masses up to 4 TeV, most recently using Run 2 data at $\sqrt{s} = 13$ TeV \cite{7tevZprime_CMSAllHad, 7tevZprime_CMSSemilept, 7tevZprime_ATLASAllHad, 7tevZprime_ATLASSemilept, 8tevZprime_CMSAllHad, 8tevZprime_CMSAllHadSemilept, 8tevZprime_CMSAllHadSemileptLept, 8tevZprime_ATLASSemilept, AN-2016-459}. The CMS and ATLAS searches are categorized by the decay of the W in the $t \rightarrow W b$ decay into all-hadronic, semi-leptonic, and fully-leptonic final states. In this analysis, we target the all-hadronic final state.

