
\vspace{-3pt}
\section{The Solenoid}\label{sec:ch3:solenoid}


The superconducting solenoid needs to be cooled to temperatures between 4.5K and 80K. To do this, liquid helium is used and the solenoid is insulated with a 40 m$^3$ vacuum chamber. The solenoid is built to withstand a misalignment up to 10 mm between the coils and the return yoke.

The CMS solenoid is 13m long with a 6m bore diameter. The tracker, ECAL, and HCAL are situated within the bore. Due to the number of turns needed to generate a 4T magnetic field, and due to the compact nature of CMS, the solenoid is wound in 4 layers.

The solenoid has a 4T magnetic field. It is 13 meters long with a 6 meter inner diameter. To keep the solenoid compact, the coil is wound 4 times over to generate the 4T magnetic field. The 4T field requires a large return yoke, and so 6 endcaps and 5 barrel wheels, weighing up to 1920 tons, make up the yoke. In order to slide the solenoid in and out of the detector, the solenoid is placed on a system of air pads and grease pads that can slide the solenoid a total of 11 meters into or out of the detector. The process of sliding the solenoid 11 meters takes 1 hour to complete.
