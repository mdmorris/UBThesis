\vspace{-3pt}
\section{The Large Hadron Collider}\label{sec:ch3:lhc}

The Large Hadron Collider (LHC) beam energy, originally at 7 TeV, now for Run 3 at 13.6 TeV, allows us to study physics at the highest energy scale in history. The collaboration also performs studies of heavy ions at 30x the energy of previous heavy ion experiments. With a luminosity for pp collisions 100x greater than previous experiments, and pp cross section of about 100 mb, measurements can be done to greater precision than ever before, and searches can probe the highest ever possible masses at the TeV scale.

The LHC contains multiple experiments. At opposite points of the collider, 27 km apart, sit the A ToroidaL ApparatuS (ATLAS) and Compact Muon Solenoid (CMS) experiments. The experiments perform similar searches and measurements without sharing preliminary results. This ensures a mitigation of biases from persons performing the analyses.

The CMS experiment has 5 layers. From innermost to outermost layer sits the tracker, the electromagnetic calorimeter, the hadronic calorimeter, the solenoid, and the muon chambers. The solenoid has a 4T magnetic field. It is 13 meters long with a 6 meter inner diameter. To keep the solenoid compact, the coil is wound 4 times over to generate the 4T magnetic field. The 4T field requires a large return yoke, and so 6 endcaps and 5 barrel wheels, weighing up to 1920 tons, make up the yoke. In order to slide the solenoid in and out of the detector, the solenoid is placed on a system of air pads and grease pads that can slide the solenoid a total of 11 meters into or out of the detector. The process of sliding the solenoid 11 meters takes 1 hour to complete.

The superconducting solenoid needs to be cooled to temperatures between 4.5K and 80K. To do this, liquid helium is used and the solenoid is insulated with a 40 m$^3$ vacuum chamber. The solenoid is built to withstand a misalignment up to 10 mm between the coils and the return yoke.
